\documentclass[a4paper]{book}
\usepackage[times,inconsolata,hyper]{Rd}
\usepackage{makeidx}
\usepackage[utf8]{inputenc} % @SET ENCODING@
% \usepackage{graphicx} % @USE GRAPHICX@
\makeindex{}
\begin{document}
\chapter*{}
\begin{center}
{\textbf{\huge rwrds}}
\par\bigskip{\large \today}
\end{center}
\begin{description}
\raggedright{}
\inputencoding{utf8}
\item[Type]\AsIs{Package}
\item[Title]\AsIs{An R package for downloading and cleaning WRDS data.}
\item[Version]\AsIs{0.1.0}
\item[Description]\AsIs{An R package for downloading and cleaning data from WRDS.}
\item[License]\AsIs{MIT + file LICENSE}
\item[Encoding]\AsIs{UTF-8}
\item[LazyData]\AsIs{true}
\item[Imports]\AsIs{dplyr,
dbplyr,
DBI,
RPostgres,
sqldf}
\item[RoxygenNote]\AsIs{6.1.1}
\end{description}
\Rdcontents{\R{} topics documented:}
\inputencoding{utf8}
\HeaderA{compustat}{Download Compustat}{compustat}
%
\begin{Description}\relax
\code{compustat} downloads the annual or quarterly Compustat table.
\end{Description}
%
\begin{Usage}
\begin{verbatim}
compustat(wrds, begin_year, end_year, frequency, vars = "default")
\end{verbatim}
\end{Usage}
%
\begin{Arguments}
\begin{ldescription}
\item[\code{wrds}] WRDS connection object from \code{wrds\_connect} function.

\item[\code{begin\_year}] Numeric.

\item[\code{end\_year}] Numeric.

\item[\code{frequency}] Character. Input either 'annual' or 'quarterly'.

\item[\code{vars}] Optional character. Defaults to a recommended subset of variables. Choose 'all' if
you want to download all variables. Otherwise, provide a list of variables.
\end{ldescription}
\end{Arguments}
%
\begin{Details}\relax
Downloads the annual or quarterly Compustat table. The table address is either compa.funda or
compa.fundq. By default, the function downloads a subset of the variables. You can specify your
own variable list by providing a string vector to the \code{vars} function. You can download all
the variables by setting \code{vars} equal to 'all'.  By default, mergeos on industry
information from the names table and removes duplicate gvkey fiscal quarter or
fiscal year observations. The de-duping logic is to keep the observation with the fewest NA
values. See function documentation for \code{compustat\_annual} and \code{compustat\_quarterly}.
\end{Details}
%
\begin{Examples}
\begin{ExampleCode}
wrds = wrds_connect(username = "testing", password = "123456")
compa_df = compustat(wrds = wrds, begin_year = 2010, end_year = 2012, frequency = 'annual', vars = c('gvkey','fyear', 'revt'))
\end{ExampleCode}
\end{Examples}
\inputencoding{utf8}
\HeaderA{compustat\_annual}{Download annual Compustat}{compustat.Rul.annual}
%
\begin{Description}\relax
\code{compustat\_annual} downloads the Compustat table with annual observations.
\end{Description}
%
\begin{Usage}
\begin{verbatim}
compustat_annual(wrds, begin_year, end_year, subset = TRUE)
\end{verbatim}
\end{Usage}
%
\begin{Arguments}
\begin{ldescription}
\item[\code{wrds}] WRDS connection object from \code{wrds\_connect} function.

\item[\code{begin\_year}] Numeric.

\item[\code{end\_year}] Numeric.

\item[\code{subset}] Optional Boolean. Download recommended subset of variables? Defaults to \code{TRUE}.
\end{ldescription}
\end{Arguments}
%
\begin{Details}\relax
Downloads the annual observation Compustat table that is updated at an annual frequency. The
table address is compa.funda. By default, the function only downloads a subset of the variables.
You can download all the variables by setting the \code{subset} argument to \code{FALSE}.
By default, merges on industry information from the names table and removes duplicate gvkey
fiscal year observations. The de-duping logic is to keep the observation with the fewest NA
values.
\end{Details}
%
\begin{Examples}
\begin{ExampleCode}
wrds = wrds_connect(username = "testing", password = "123456")
compa_df = compustat_annual(wrds = wrds, begin_year = 2010, end_year = 2012)
\end{ExampleCode}
\end{Examples}
\inputencoding{utf8}
\HeaderA{compustat\_append\_crsp\_links}{Append CRSP identifiers onto a Compustat data frame}{compustat.Rul.append.Rul.crsp.Rul.links}
%
\begin{Description}\relax
\code{compustat\_append\_crsp\_links} appends on CRSP permno and permco identifiers onto Compustat
gvkey identifiers.
\end{Description}
%
\begin{Usage}
\begin{verbatim}
compustat_append_crsp_links(wrds, comp_df)
\end{verbatim}
\end{Usage}
%
\begin{Arguments}
\begin{ldescription}
\item[\code{wrds}] WRDS connection object from \code{wrds\_connect} function.

\item[\code{comp\_df}] Compustat data.frame. Requires the gkvey and datadate fields to be present.
\end{ldescription}
\end{Arguments}
%
\begin{Details}\relax
Appends on CRSP permno and permcos based on the standard Compsutat-CRSP linking table. The
linking table address is crspa.ccmxpf\_linktable. Requires that the Compustat data.frame be
physically located in memory (not a lazy \code{dplyr} table reference). Requires that the gvkey
and datadate fields are present in the Compustat data.frame.
\end{Details}
%
\begin{Examples}
\begin{ExampleCode}
wrds = wrds_connect(username = "testing", password = "123456")
compa_annual = compustat_annual(wrds = wrds, begin_year = 2010, end_year = 2011)
comp_linked = compustat_append_crsp_links(wrds = wrds, comp_df = compa_annual)
\end{ExampleCode}
\end{Examples}
\inputencoding{utf8}
\HeaderA{compustat\_crsp\_annual}{Download annual Compustat with CRSP annual returns}{compustat.Rul.crsp.Rul.annual}
%
\begin{Description}\relax
\code{compustat\_annual} downloads the annual Compustat table and merges on annual CRSP returns.
\end{Description}
%
\begin{Usage}
\begin{verbatim}
compustat_crsp_annual(wrds, begin_year, end_year, subset = TRUE)
\end{verbatim}
\end{Usage}
%
\begin{Arguments}
\begin{ldescription}
\item[\code{wrds}] WRDS connection object from \code{wrds\_connect} function.

\item[\code{begin\_year}] Numeric.

\item[\code{end\_year}] Numeric.

\item[\code{subset}] Optional Boolean. Download recommended subset of variables? Defaults to \code{TRUE}.
\end{ldescription}
\end{Arguments}
%
\begin{Details}\relax
Downloads the annual Compustat file using the \code{compustat\_annual} function. Appends on
CRSP links and returns using the functions \code{compustat\_append\_crsp\_links} and
\code{crsp\_annual}, respectively. The annual return merge is naive and is based on a fiscal year
to calendar year match.
\end{Details}
%
\begin{Examples}
\begin{ExampleCode}
wrds = wrds_connect(username = "testing", password = "123456")
comp_crsp_df = compustat_crsp_annual(wrds = wrds, begin_year = 2010, end_year = 2012)
\end{ExampleCode}
\end{Examples}
\inputencoding{utf8}
\HeaderA{compustat\_crsp\_linking\_table}{Download Compustat-CRSP linking table}{compustat.Rul.crsp.Rul.linking.Rul.table}
%
\begin{Description}\relax
\code{compustat\_crsp\_linking\_table} downloads the Compustat-CRSP linking table.
\end{Description}
%
\begin{Usage}
\begin{verbatim}
compustat_crsp_linking_table(wrds, dl = TRUE)
\end{verbatim}
\end{Usage}
%
\begin{Arguments}
\begin{ldescription}
\item[\code{wrds}] WRDS connection object from \code{wrds\_connect} function.

\item[\code{dl}] Optional Boolean. Download the data? Defaults to \code{TRUE}. \code{FALSE} outputs a
lazy \code{dplyr} table reference.
\end{ldescription}
\end{Arguments}
%
\begin{Details}\relax
Downloads the standard Compustat-CRSP linking table. The table address is crspa.ccmxpf\_linktable.
By default, the function downloads the data. A lazy \code{dplyr} table reference can be returned
by setting the \code{dl} argument to \code{FALSE}.
\end{Details}
%
\begin{Examples}
\begin{ExampleCode}
wrds = wrds_connect(username = "testing", password = "123456")
linking_tbl = compustat_crsp_linking_table(wrds = wrds, dl = FALSE)
\end{ExampleCode}
\end{Examples}
\inputencoding{utf8}
\HeaderA{compustat\_names}{Download Compustat names table}{compustat.Rul.names}
%
\begin{Description}\relax
\code{compustat\_names} downloads the Compustat names table.
\end{Description}
%
\begin{Usage}
\begin{verbatim}
compustat_names(wrds, subset = TRUE, dl = TRUE)
\end{verbatim}
\end{Usage}
%
\begin{Arguments}
\begin{ldescription}
\item[\code{wrds}] WRDS connection object from \code{wrds\_connect} function.

\item[\code{subset}] Optional Boolean. Download recommended subset of variables? Defaults to \code{TRUE}.

\item[\code{dl}] Optional Boolean. Download the data? Defaults to \code{TRUE}. \code{FALSE} outputs a
lazy \code{dplyr} table reference.
\end{ldescription}
\end{Arguments}
%
\begin{Details}\relax
The names table contains general gvkey-level information, such as NAICS industry. The table
address is compa.names. By default, the function only downloads a subset of the variables.
You can download all the variables by setting the \code{subset} argument to \code{FALSE}.
\end{Details}
%
\begin{Examples}
\begin{ExampleCode}
wrds = wrds_connect(username = "testing", password = "123456")
names_df = compustat_names(wrds = wrds, subset = TRUE, dl = TRUE)
\end{ExampleCode}
\end{Examples}
\inputencoding{utf8}
\HeaderA{crsp}{Download CRSP}{crsp}
%
\begin{Description}\relax
\code{crsp} downloads the annual, monthly, or daily CRSP table.
\end{Description}
%
\begin{Usage}
\begin{verbatim}
crsp(wrds, begin_year, end_year, frequency, dl = FALSE)
\end{verbatim}
\end{Usage}
%
\begin{Arguments}
\begin{ldescription}
\item[\code{wrds}] WRDS connection object from \code{wrds\_connect} function.

\item[\code{begin\_year}] Numeric.

\item[\code{end\_year}] Numeric.

\item[\code{frequency}] Character. Input either 'annual', 'monthly', or 'daily'.

\item[\code{dl}] Optional Boolean. Download the data? Defaults to \code{TRUE}. \code{FALSE} outputs a
lazy \code{dplyr} table reference.
\end{ldescription}
\end{Arguments}
%
\begin{Details}\relax
Downloads the annual, monthly, or daily CRSP Compustat table. The schema is crspa and the
table is either the aggregated msf, msf, or dsf. Removes observations with weakly negative prices
and shares outstanding. The function merges on identifying information from the
CRSP header table. Please see either \code{crsp\_annual}, \code{crsp\_monthly}, or
\code{crsp\_daily} documentation for more details.
\end{Details}
%
\begin{Examples}
\begin{ExampleCode}
wrds = wrds_connect(username = "testing", password = "123456")
crsp_df = crsp(wrds = wrds, begin_year = 2010, end_year = 2012, frequency = 'annual')
\end{ExampleCode}
\end{Examples}
\inputencoding{utf8}
\HeaderA{crsp\_annual}{Download CRSP annual returns and end-of-year prices}{crsp.Rul.annual}
%
\begin{Description}\relax
\code{crsp\_annual} downloads CRSP annual returns and end-of-year prices.
\end{Description}
%
\begin{Usage}
\begin{verbatim}
crsp_annual(wrds, begin_year, end_year, dl = TRUE)
\end{verbatim}
\end{Usage}
%
\begin{Arguments}
\begin{ldescription}
\item[\code{wrds}] WRDS connection object from \code{wrds\_connect} function.

\item[\code{begin\_year}] Numeric.

\item[\code{end\_year}] Numeric.

\item[\code{dl}] Optional Boolean. Download the data? Defaults to \code{TRUE}. \code{FALSE} outputs a
lazy \code{dplyr} table reference.
\end{ldescription}
\end{Arguments}
%
\begin{Details}\relax
Constructs annual returns from the monthly stock file, located at crspa.msf. Only keeps permnos
with a full year of data with non-missing values for returns and strictly positive values
for shares outstanding and price. The function merges on identifying information from the
CRSP header table.
\end{Details}
%
\begin{Examples}
\begin{ExampleCode}
wrds = wrds_connect(username = "testing", password = "123456")
crspa_df = crsp_annual(wrds = wrds, begin_year = 2010, end_year = 2012)
\end{ExampleCode}
\end{Examples}
\inputencoding{utf8}
\HeaderA{crsp\_daily}{Download CRSP daily return and price data}{crsp.Rul.daily}
%
\begin{Description}\relax
\code{crsp\_daily} downloads CRSP daily returns and end-of-month prices.
\end{Description}
%
\begin{Usage}
\begin{verbatim}
crsp_daily(wrds, begin_year, end_year, dl = TRUE)
\end{verbatim}
\end{Usage}
%
\begin{Arguments}
\begin{ldescription}
\item[\code{wrds}] WRDS connection object from \code{wrds\_connect} function.

\item[\code{begin\_year}] Numeric.

\item[\code{end\_year}] Numeric.

\item[\code{dl}] Optional Boolean. Download the data? Defaults to \code{TRUE}. \code{FALSE} outputs a
lazy \code{dplyr} table reference.

\item[\code{subset}] Optional Boolean. Download recommended subset of variables? Defaults to \code{TRUE}.
\end{ldescription}
\end{Arguments}
%
\begin{Details}\relax
Downloads data from table crspa.dsf. Removes observations with weakly negative prices and
shares outstanding. The function merges on identifying information from the
CRSP header table.
\end{Details}
%
\begin{Examples}
\begin{ExampleCode}
wrds = wrds_connect(username = "testing", password = "123456")
crspd_df = crsp_daily(wrds = wrds, begin_year = 2010, end_year = 2010)
\end{ExampleCode}
\end{Examples}
\inputencoding{utf8}
\HeaderA{crsp\_header}{Download CRSP header table}{crsp.Rul.header}
%
\begin{Description}\relax
\code{crsp\_header} downloads the CRSP header table.
\end{Description}
%
\begin{Usage}
\begin{verbatim}
crsp_header(wrds, subset = TRUE, dl = TRUE)
\end{verbatim}
\end{Usage}
%
\begin{Arguments}
\begin{ldescription}
\item[\code{wrds}] WRDS connection object from \code{wrds\_connect} function.

\item[\code{subset}] Optional Boolean. Download recommended subset of variables? Defaults to \code{TRUE}.

\item[\code{dl}] Optional Boolean. Download the data? Defaults to \code{TRUE}. \code{FALSE} outputs a
lazy \code{dplyr} table reference.
\end{ldescription}
\end{Arguments}
%
\begin{Details}\relax
The header table contains permno-level information, such as SIC industry. The table
address is crspa.msfhdr. By default, the function only downloads a subset of the variables.
You can download all the variables by setting the \code{subset} argument to \code{FALSE}.
\end{Details}
%
\begin{Examples}
\begin{ExampleCode}
wrds = wrds_connect(username = "testing", password = "123456")
crsp_names = crsp_header(wrds = wrds, subset = TRUE, dl = FALSE)
\end{ExampleCode}
\end{Examples}
\inputencoding{utf8}
\HeaderA{crsp\_monthly}{Download CRSP monthly return and price data}{crsp.Rul.monthly}
%
\begin{Description}\relax
\code{crsp\_monthly} downloads CRSP monthly returns and end-of-month prices.
\end{Description}
%
\begin{Usage}
\begin{verbatim}
crsp_monthly(wrds, begin_year, end_year, dl = TRUE)
\end{verbatim}
\end{Usage}
%
\begin{Arguments}
\begin{ldescription}
\item[\code{wrds}] WRDS connection object from \code{wrds\_connect} function.

\item[\code{begin\_year}] Numeric.

\item[\code{end\_year}] Numeric.

\item[\code{dl}] Optional Boolean. Download the data? Defaults to \code{TRUE}. \code{FALSE} outputs a
lazy \code{dplyr} table reference.
\end{ldescription}
\end{Arguments}
%
\begin{Details}\relax
Downloads data from table crspa.msf. Removes observations with weakly negative prices and
shares outstanding. The function merges on identifying information from the
CRSP header table.
\end{Details}
%
\begin{Examples}
\begin{ExampleCode}
wrds = wrds_connect(username = "testing", password = "123456")
crspm_df = crsp_monthly(wrds = wrds, begin_year = 2010, end_year = 2012)
\end{ExampleCode}
\end{Examples}
\inputencoding{utf8}
\HeaderA{hello}{Hello, World!}{hello}
%
\begin{Description}\relax
Prints 'Hello, world!'.
\end{Description}
%
\begin{Usage}
\begin{verbatim}
hello()
\end{verbatim}
\end{Usage}
%
\begin{Examples}
\begin{ExampleCode}
hello()
\end{ExampleCode}
\end{Examples}
\inputencoding{utf8}
\HeaderA{mergent\_corporates}{Download Mergent corporate bonds}{mergent.Rul.corporates}
%
\begin{Description}\relax
\code{mergent\_corporates} downloads corporate bond issues from the Mergent issues table
\end{Description}
%
\begin{Usage}
\begin{verbatim}
mergent_corporates(wrds, clean = TRUE, vanilla = TRUE, subset = TRUE,
  dl = TRUE)
\end{verbatim}
\end{Usage}
%
\begin{Arguments}
\begin{ldescription}
\item[\code{wrds}] WRDS connection object from \code{wrds\_connect} function.

\item[\code{clean}] Optional Boolean. Clean the data to remove data errors and missings? Defaults to
\code{TRUE}.

\item[\code{vanilla}] Optional Boolean. Restrict data to common vanilla bonds? Defaults to \code{TRUE}.

\item[\code{subset}] Optional Boolean. Clean the data for errors and missing fields? Defaults to
\code{TRUE}.

\item[\code{dl}] Optional Boolean. Download the data? Defaults to \code{TRUE}. \code{FALSE} outputs a
lazy \code{dplyr} table reference.
\end{ldescription}
\end{Arguments}
%
\begin{Details}\relax
Downloads issue-level bond data from the table fisd.fisd\_mergedissue for corporate bonds.
Corporates are defined by being in utility, financial, or industrial Mergent industry groups,
having bond types in CDEB, CMTN, CZ, CMTZ, UCID, or CP, and having security levels in
SEN, SENS, SUB, or SS. This function calls the \code{mergent\_issues} function to download the
data You can download all the variables by setting the \code{subset}
argument to \code{FALSE}. By default, this function also cleans the data and restricts to
vanilla corporates, which removes: floating rates notes, convertible bonds, foreign domiciled
issuers, foreign currency notes, private placements, perpetuals, preferred securities,
retail notes, slobs, exchangeables, unit deals, pay-in-kinds, defeased bonds, and bonds with
non-strictly positive offering amounts and missing coupon types, issue ids, maturity, and
offering date. By default, also merges on initial credit ratings for the issue and makes
simple variable transformations.
\end{Details}
%
\begin{Examples}
\begin{ExampleCode}
wrds = wrds_connect(username = "testing", password = "123456")
mergent_corps = mergent_corporates(wrds, clean = TRUE, vanilla = TRUE, subset = TRUE, dl = FALSE)
\end{ExampleCode}
\end{Examples}
\inputencoding{utf8}
\HeaderA{mergent\_issues}{Download Mergent bond issues table}{mergent.Rul.issues}
%
\begin{Description}\relax
\code{mergent\_issues} downloads the Mergent bond issues table.
\end{Description}
%
\begin{Usage}
\begin{verbatim}
mergent_issues(wrds, clean = TRUE, vanilla = TRUE, subset = TRUE,
  dl = TRUE)
\end{verbatim}
\end{Usage}
%
\begin{Arguments}
\begin{ldescription}
\item[\code{wrds}] WRDS connection object from \code{wrds\_connect} function.

\item[\code{clean}] Optional Boolean. Clean the data to remove data errors and missings? Defaults to
\code{TRUE}.

\item[\code{vanilla}] Optional Boolean. Restrict data to common vanilla bonds? Defaults to \code{TRUE}.

\item[\code{subset}] Optional Boolean. Clean the data for errors and missing fields? Defaults to
\code{TRUE}.

\item[\code{dl}] Optional Boolean. Download the data? Defaults to \code{TRUE}. \code{FALSE} outputs a
lazy \code{dplyr} table reference.
\end{ldescription}
\end{Arguments}
%
\begin{Details}\relax
Downloads issue-level bond data from the table fisd.fisd\_mergedissue. Appends on information
from the callable, agentid, and exchangeable tables. By default, the function only downloads a
subset of the variables. You can download all the variables by setting the \code{subset}
argument to \code{FALSE}. By default, this function also cleans the data and restricts to
vanilla bonds, which removes: floating rates notes, convertible bonds, foreign domiciled
issuers, foreign currency notes, private placements, perpetuals, preferred securities,
retail notes, slobs, exchangeables, unit deals, pay-in-kinds, defeased bonds, and bonds with
non-strictly positive offering amounts and missing coupon types, issue ids, maturity, and
offering date. By default, also merges on initial credit ratings for the issue and makes
simple variable transformations.
\end{Details}
%
\begin{Examples}
\begin{ExampleCode}
wrds = wrds_connect(username = "testing", password = "123456")
mergent_issues_df = mergent_issues(wrds, clean = TRUE, vanilla = TRUE, subset = TRUE, dl = FALSE)
\end{ExampleCode}
\end{Examples}
\inputencoding{utf8}
\HeaderA{wrds\_connect}{Connect to the WRDS cloud}{wrds.Rul.connect}
%
\begin{Description}\relax
\code{wrds\_connect} creates a remote connection to the WRDS cloud.
\end{Description}
%
\begin{Usage}
\begin{verbatim}
wrds_connect(username, password)
\end{verbatim}
\end{Usage}
%
\begin{Arguments}
\begin{ldescription}
\item[\code{username}] Character. WRDS username.

\item[\code{password}] Character. WRDS password.
\end{ldescription}
\end{Arguments}
%
\begin{Details}\relax
Note: WRDS cloud runs PostgreSQL.
\end{Details}
%
\begin{Examples}
\begin{ExampleCode}
wrds = wrds_connect(username = Sys.getenv("WRDS_NAME"), password = Sys.getenv("WRDS_PASS"))
\end{ExampleCode}
\end{Examples}
\inputencoding{utf8}
\HeaderA{wrds\_schema\_list}{List available schemas in WRDS}{wrds.Rul.schema.Rul.list}
%
\begin{Description}\relax
\code{wrds\_schema\_list} downloads a list of the available schemas in the WRDS cloud.
\end{Description}
%
\begin{Usage}
\begin{verbatim}
wrds_schema_list(wrds)
\end{verbatim}
\end{Usage}
%
\begin{Arguments}
\begin{ldescription}
\item[\code{wrds}] WRDS connection object from \code{wrds\_connect} function.
\end{ldescription}
\end{Arguments}
%
\begin{Details}\relax
Schemas represent different data vendors in WRDS. Schemas can be thought of as separate
databases that contain their own set of tables.
\end{Details}
%
\begin{Examples}
\begin{ExampleCode}
wrds = wrds_connect(username = "testing", password = "123456")
schema_list = wrds_schema_list(wrds = wrds)
\end{ExampleCode}
\end{Examples}
\inputencoding{utf8}
\HeaderA{wrds\_table\_list}{List the tables in a WRDS schema.}{wrds.Rul.table.Rul.list}
%
\begin{Description}\relax
\code{wrds\_table\_list} lists the tables in a WRDS schema
\end{Description}
%
\begin{Usage}
\begin{verbatim}
wrds_table_list(wrds, schema)
\end{verbatim}
\end{Usage}
%
\begin{Arguments}
\begin{ldescription}
\item[\code{wrds}] WRDS connection object from \code{wrds\_connect} function.

\item[\code{schema}] Character. WRDS schema from \code{wrds\_schema\_list}.
\end{ldescription}
\end{Arguments}
%
\begin{Details}\relax
Each WRDS schema contains its own set of tables. Tables are queried directly in the package.
\end{Details}
%
\begin{Examples}
\begin{ExampleCode}
wrds = wrds_connect(username = "testing", password = "123456")
table_list = wrds_table_list(wrds = wrds, schema = "compa")
\end{ExampleCode}
\end{Examples}
\inputencoding{utf8}
\HeaderA{wrds\_variable\_list}{List the variables in a WRDS table}{wrds.Rul.variable.Rul.list}
%
\begin{Description}\relax
\code{wrds\_variable\_list} lists the columns (variables) in a WRDS table.
\end{Description}
%
\begin{Usage}
\begin{verbatim}
wrds_variable_list(wrds, schema, table)
\end{verbatim}
\end{Usage}
%
\begin{Arguments}
\begin{ldescription}
\item[\code{wrds}] WRDS connection object from \code{wrds\_connect} function.

\item[\code{schema}] Character. WRDS schema from \code{wrds\_schema\_list}.

\item[\code{table}] Character. WRDS table from \code{wrds\_table\_list}.
\end{ldescription}
\end{Arguments}
%
\begin{Details}\relax
NA.
\end{Details}
%
\begin{Examples}
\begin{ExampleCode}
wrds = wrds_connect(username = "testing", password = "123456")
variable_list = wrds_variable_list(wrds = wrds, schema = "compa", table = "funda")
\end{ExampleCode}
\end{Examples}
\printindex{}
\end{document}
